\documentclass[10pt]{article}

\usepackage{graphicx}
\usepackage{amsmath,amsfonts,amssymb}

% use different colors for links:
\usepackage{color}
\definecolor{darkgreen}{rgb}{0.1,0.5,0.1}
\definecolor{darkblue}{rgb}{0.2,0.2,1.0}
\usepackage[colorlinks=true,linkcolor=darkblue,citecolor=darkblue,
            filecolor=darkblue,urlcolor=darkgreen]{hyperref}


\setlength{\textwidth}{6.2in}
\setlength{\oddsidemargin}{0.3in}
\setlength{\evensidemargin}{0in}
\setlength{\textheight}{8.9in}
\setlength{\voffset}{-1in}
\setlength{\headsep}{26pt}
\setlength{\parindent}{0pt}
\setlength{\parskip}{5pt}



\input{../latex/macros.tex}  % input some useful macros

\begin{document}

% header:
\hfill\vbox{\hbox{AMath 586 / ATM 581}
\hbox{Homework \#5}\hbox{Due Tuesday, June 4, 2019}}

{\bf Name:} Your name here
\vskip 5pt

Homework is due to Canvas by 11:00pm PDT on the due date.

To submit, see \url{https://canvas.uw.edu/courses/1271892/assignments/4825694}


%--------------------------------------------------------------------------
\vskip 1cm
\hrule
{\bf Problem 1}  

Let $U = [U_0,~U_1,~\ldots,~U_m]^T$ be a vector of function values at
equally spaced points on the interval $0\leq x \leq 1$, and suppose the
underlying function is periodic and smooth.  Then we can approximate 
the first derivative $u_x$ at all of these points by $DU$, where $D$ is
circulant matrix such as
\[
D_- = \frac 1 h \brm 1&&&&-1\\ -1&1\\ &-1&1\\ &&-1&1\\ &&&-1&1\erm,  \qquad
D_+ = \frac 1 h \brm -1&1\\ &-1&1\\ &&-1&1\\ &&&-1&1\\ 1&&&&-1\erm
\]
for first-order accurate one-sided approximations, or
\[
D_0 = \frac 1 {2h} \brm 0&1&&&-1\\ -1&0&1\\ &-1&0&1\\ &&-1&0&1\\ 1&&&-1&0\erm
\]
for a second-order accurate centered approximation.  (These are illustrated
for a grid with $m+1=5$ unknowns and $h=1/5$.)


The advection equation $u_t + au_x=0$ on the interval $0\leq x \leq
1$ with periodic boundary conditions  
gives rise to the MOL discretization $U'(t) = -aDU(t)$
where $D$ is one of the matrices above.

\begin{enumerate} 
\item Discretizing $U' = -aD_-U$ by forward Euler gives the first order
upwind method
\[
U_j^{n+1} = U_j^n - \frac{ak}{h} (U_j^n - U_{j-1}^n),
\]
where the index $i$ runs from 0 to $m$ with addition of indices performed
mod $m+1$ to incorporate the periodic boundary conditions.

Suppose instead we discretize the MOL equation by the second-order Taylor
series method, 
\[
U^{n+1} = U^n - akD_-U^n + \half (ak)^2 D_-^2 U^n.
\]
Compute $D_-^2$ and also write out the formula for $U_j^n$ that results from
this method.  

\item How accurate is the method derived in part (a) compared to the
Beam-Warming method, which is also a 3-point one-sided method?

\item Suppose we make the method more symmetric:
\[
U^{n+1} = U^n - \frac{ak}{2} (D_+ +D_-)U^n + \half (ak)^2 D_+D_- U^n.
\]
Write out the formula for $U_j^n$ that results from this method.  
What standard method is this?

\end{enumerate}



% uncomment the next two lines if you want to insert solution...
%\vskip 1cm
%{\bf Solution:}

% insert your solution here!

%--------------------------------------------------------------------------
\vskip 1cm
\hrule
{\bf Problem 2}  


\begin{enumerate} 
\item
Produce a plot similar to those shown in Figure 10.1 for the upwind method
(10.21) with the same values of $a=1$, $h=1/50$ and $k=0.8h$
used in that figure.

\item Produce the corresponding plot if the one-sided method (10.22) is
instead used with the same values of $a,~h$, and $k$ as in part (a).
Comment on how this relates to what stability theory tells you about this
method with these parameters.
\end{enumerate}

% uncomment the next two lines if you want to insert solution...
%\vskip 1cm
%{\bf Solution:}

% insert your solution here!
%--------------------------------------------------------------------------

\vskip 1cm
\hrule
{\bf Problem 3}  

The notebook \verb+notebooks/Advection.ipynb+ illustrates several numerical
methods for the advection equation, with periodic boundary conditions.

Familiarize yourself with how this works by experimenting with
different parameter choices.  In particular understand how the periodic
boundary conditions are implemented.

\begin{enumerate}
\item Modify the notebook to also implement the Beam-Warming method from
Section 10.4.2.  Check that you observe second order accuracy.  Make sure you
implement the correct version based on the sign of $a$.  (Assume $a>0$.)

\item
Experiment with different stepsizes to test the expected stability limit
of this method.  What happens if it the time step is chosen to be
right at the stability limit?  Why?

\item
Do the oscillations that develop trail behind the Gaussian peak or lead ahead
of it?  You should be able to observe both cases depending on how you choose
the time step.  Confirm that what you see agrees with what is expected 
from Example 10.11, and produce sample plots for each case.

\end{enumerate}

% uncomment the next two lines if you want to insert solution...
%\vskip 1cm
%{\bf Solution:}

% insert your solution here!


%--------------------------------------------------------------------------

\end{document}

